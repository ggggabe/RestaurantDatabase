\documentclass[11pt]{article}

\topmargin -.5in
\textheight 9in
\oddsidemargin 14pt
\evensidemargin 14pt
\textwidth 6in

\input{/home/kevinsung/Dropbox/Academic/latexdefs}

\begin{document}

\begin{center}
    \textbf{CS 336 Project Description} \\
    Gabe Marquez and Kevin J. Sung
\end{center}

\begin{flushleft}
    Scheme: Food\\
    Audience: Restaurant owners\\
    Goal: Help restaurants raise customer satisfaction and profits\\
\end{flushleft}

In addition to the core scheme, we have unique data about patrons' dietary restrictions
(specifically, which ingredients they never consume),
ingredients contained in items served by restaurants, and information associated
with specific orders.

Key to our database is the relation "Order", which contains the key "order\_id".
With each order, we associate a patron, date, restaurant, and tip amount. Through
the attribute "order\_id", we can access the table "Ordered", which lists
the items associated with each order, i.e., what the unique patron associated with
that order\_id ordered during that visit.

To use this information to help restaurants maximize profit, we introduce the
"Customer Satisfaction Metric" (CSM), which is a number associated with each restaurant
on each day that is a measure of how satisfied its customers were on that day.
The CSM is a function of the size of the tips received at the restaurant, the
number of customers the restaurant had, the amount of money customers spent, and
the options that customers had when eating at that restaurant (i.e., if the restaurant
serves mainly dishes that contain meat that day, then vegetarians who visited the restaurant
that day may not have had many options, and hence lowering the CSM). We look for
trends in the CSM that will indicate restaurant practices which result in higher CSM,
and hence more happy and loyal customers and higher profits.

\end{document}
